%% td6.tex

\documentclass[a4paper,10pt]{article}
\usepackage[latin1]{inputenc}
\usepackage[francais]{babel}
\usepackage{fullpage}

\usepackage{moreverb}
\usepackage{alltt}
\usepackage{amsmath}
\usepackage{amssymb}
\usepackage{pifont}
\usepackage{longtable}
\usepackage[all]{xy}
\usepackage{textcomp}
\usepackage{color}
\usepackage{longtable}
\usepackage{placeins}
\usepackage{txfonts}

\newcommand{\unix}{{\sc Unix}}

\newtheorem{exercice}{Exercice}

\title{CCI - Principes des langages de programmation\\ TD 6 :
  Structures avanc\'{e}es, pointeurs et tri}

\date{12 octobre 2006}

\author{Nicolas Bredeche\footnote{bredeche@lri.fr}, Thomas Tang\footnote{ttang@club-internet.fr}}

\begin{document}

\maketitle

\section{Pour bien assimiler...}

\subsection{Tableaux et Pointeurs}

\begin{exercice}
Soit le programme suivant :
  \begin{listing}{0}
  int main(){
    int i = 1 ;
    int * pt ;
    pt = malloc (sizeof(int));
    if (pt != NULL){
        *pt = i ;
        printf("i = %i\t*pt = %i\n",i,*pt) ;
        *pt = 2 ;
        printf("i = %i\t*pt = %i\n",i,*pt) ;
        pt = &i ;
        *pt = 3 ;
        printf("i = %i\t*pt = %i\n",i,*pt) ;
        exit(0) ;
    }
    else{
        printf("Probl\`{e}me d'allocation de la m\'{e}moire\n");
        exit(-1);
    }
  }
  \end{listing}

  La valeur affich\'{e}e de i change-t-elle lors du deuxi\`{e}me affichage ?
  Et lors du troisi\`{e}me affichage ? Pourquoi ?

\end{exercice}

En langage C, un tableau est en fait un pointeur sur le premier
\'{e}l\'{e}ment du tableau, l'acc\`{e}s au $i$-i\`{e}me
\'{e}l\'{e}ment se fait en ajoutant $i$ \`{a} l'adresse de ce
pointeur. Notez que les cha�nes de caract\`{e}res sont en fait des
tableaux (une cha�ne d\'{e}finit avec {\tt char *} est en
fait un pointeur sur le premier caract\`{e}re de la chaine.\\
\\
On peut se d\'{e}placer dans un tableau \`{a} l'aide de l'addition
sur les pointeurs. Concr\`{e}tement, si {\tt tab} est un tableau
d'entiers, l'expression {\tt *(tab + i)} est \'{e}quivalente \`{a}
{\tt tab[i]}

\begin{exercice}
\`{A} l'aide de l'addition sur les pointeurs, utilisez un pointeur
pour remplir et afficher un tableau des N premiers entiers, avec N
demand\'{e} \`{a} l'utilisateur. Puis remplir et afficher un tableau
de N caract\`{e}res avec la lettre 'i' (resp. 'p') pour toutes les
positions impaires (resp. paires).
\end{exercice}

\begin{exercice}
  \'{E}crire un programme qui \`{a} l'aide d'une boucle affiche tous les
  caract\`{e}res de 0 \`{a} 255 ainsi que leur num\'{e}ro.
\end{exercice}

\begin{exercice}
  Remplir un tableau avec l'alphabet en minuscules (utilisez l'affichage de l'exercice pr\'{e}c\'{e}dent pour
  conna�tre les caract\`{e}res correspondants).
  Passer ensuite tous les caract\`{e}res en majuscules.
\end{exercice}

\subsection{Tris}

\begin{exercice}
  Dans cette exercice on se propose de trier un tableau \`{a} l'aide de
  deux pointeurs. Le but est de classer les entiers contenu dans le
  tableau par ordre croissant. On va trier le tableau par la fin, et pour
  ce faire, on a besoin d'un pointeur sur le dernier \'{e}l\'{e}ment du tableau ({\tt
  dernier = tab + (taille - 1)}). On commence par chercher le plus
  grand \'{e}l\'{e}ment du tableau et on l'\'{e}change avec le dernier \'{e}l\'{e}ment,
  puis on recommence en d\'{e}calant le dernier \'{e}l\'{e}ment ({\tt dernier =
  dernier - 1}) jusqu'\`{a} ce que le tableau soit tri\'{e}.\\
  \\
  Cet exercice sera fait en utilisant d'une part la notation tab[i],
  et d'autre part avec des pointeurs.
\end{exercice}

\subsection{Structures et structures r\'{e}cursives}

\begin{exercice}
  Vous d\'{e}finirez deux structures {\tt personne} et {\tt adresse} li\'{e}es
  entre elles (une personne habite \`{a} une adresse.) Votre structure
  personne contiendra \'{e}galement les liens de filiations (p\`{e}re et
  m\`{e}re.) Dans votre programme vous cr\'{e}erez 5 personnes dont 3
  habitent \`{a} la m\^{e}me adresse et 2 ont les m\^{e}mes parents. Utilisez des
  pointeurs dans vos structures pour qu'une m\^{e}me adresse n'existe pas
  plusieurs fois en m\'{e}moire. Pr\'{e}voir une fonction qui affiche une
  personne et l'appeler pour toutes les personnes d\'{e}finies.
\end{exercice}

\section{Pour bien approfondir...}

\begin{exercice}
Algorithme de classement des valeurs d'un tableau dans l'ordre
croissant par recherche de minimum (tri par extraction)
\begin{enumerate}
\item Etat initial du tableau, recherche sur l'ensemble du tableau de
la plus petite valeur et permutation avec le premier
\'{e}l\'{e}ment.
\item Le traitement pr\'{e}c\'{e}dent est de nouveau appliqu\'{e} mais sur le
tableau r\'{e}duit du premier \'{e}l\'{e}ment d\'{e}j\`{a}
plac\'{e}.
\item Le traitement
est identique avec \`{a} chaque permutation le tableau r\'{e}duit
d'un \'{e}l\'{e}ment. le traitment s'arr\^{e}te lorsque le nombre
d'\'{e}l\'{e}ments restant \`{a} classer est \'{e}gal \`{a} 1.
\end{enumerate}
Commencez \`{a} trier \`{a} la main la liste suivant en respectant
les r\`{e}gles de l'algorithme: (5,4,7,10,1)\\
Ecrivez une fonction \textbf{void tri\_extraction (int
*tab)} qui r\'{e}alise ce tri.
\end{exercice}


\begin{exercice}
Algorithme de classement des valeurs d'un tableau par permutation et
retour en arri\`{e}re (tri par bulles ou par \'{e}change)
\begin{enumerate}
\item Comparer deux \'{e}l\'{e}ments voisins
\item Si les \'{e}l\'{e}ments sont class\'{e}s, passage \`{a} l'\'{e}l\'{e}ment suivant
(sauf au bas du tableau)
\item Si les \'{e}l\'{e}ments ne sont pas class\'{e}s, permutez les deux
\'{e}l\'{e}ments, et revenez vers l'\'{e}l\'{e}ment
pr\'{e}c\'{e}dent (sauf en haut du tableau o� l'on passe \`{a}
l'\'{e}l\'{e}ment suivant)
\item Le traitement s'arr\^{e}te quand la comparaison des deux \'{e}l\'{e}ments
du bas de tableau n'entra�ne pas de permutation.
\end{enumerate}
Commencez \`{a} trier \`{a} la main la liste suivant en respectant
les r\`{e}gles de l'algorithme: (5,4,7,10,1)\\
Ecrivez une fonction \textbf{void tri\_bulles (int
*tab)} qui r\'{e}alise ce tri.
\end{exercice}


\begin{exercice}
Listes cha�n\'{e}es\\
\\
Une liste chain\'{e}e est form\'{e}e de cellules li\'{e}es les unes
aux autres par des pointeurs. On a une cellule ent\^{e}te et une
cellule queue. Chaque cellule de la liste est de type :
\begin{listing}{0}
struct cellule{
    int id;                 //information correspondant \`{a} la cellule
    struct cellule *suivant //pointeur vers la cellule suivante de
    la liste
};
\end{listing}
Cr\'{e}ez une liste cha�n\'{e}e de 10 \'{e}l\'{e}ments, o� l'entier
de chaque cellule est demand\'{e} \`{a} l'utilisateur. Affichez
ensuite cette cha�ne.
\end{exercice}

\end{document}
