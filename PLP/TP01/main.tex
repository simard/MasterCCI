\documentclass[a4paper]{article}
	\usepackage{Lapin}
	\newenvironment{Correction}{\par\tiny\blue\rule[1ex]{\textwidth}{1pt}\par\normalsize\textbf{\sffamily{}Correction de l'exercice~\theExo{} -- }}{\par\tiny\blue\rule[1ex]{\textwidth}{1pt}\par}
	\newtheorem{Exo}{{\sffamily{Exercice}}}
	\title{Introduction}
	\author{Philippe \Nom{Rinaudo} \and{} Jean \Nom{Simard}}
	\date{\Date[l]{01}{10}{2009}}
\begin{document}
	\maketitle
	\section{Commencer avec un terminal}
		Le terminal de commandes va vous permettre d'effectuer toutes les opérations nécessaires pour utiliser un ordinateur et en particulier la programmation.
		Nous allons donc simplement rappeler les quelques commandes simples pour travailler sur un terminal.
		\subsection{\code{ls}}
			La commande \code{ls} permet de lister le contenu d'un répertoire, de connaître le contenu.
			Quelques options additionnelles peuvent s'avérer utiles.
			Si vous tapez \code{ls -l}, vous obtiendrez des informations supplémentaires.
			En tapant \code{ls -a}, les fichiers et répertoires cachés seront également listés
			Les options peuvent être combinés.
			Par exemple, \code{ls -alh} vous permettra de lister tous les fichier et répertoires, même ceux cachés, de façon détaillée et en affichant la taille des fichiers.

			Pour vous rappeler de cette commande, \code{ls} sont les lettres utilisées dans \emph{LiSt}.
		\subsection{\code{cd}}
			Cette commande \code{cd} va vous permettre de vous déplacer dans les différents répertoires.
			\code{cd} correspond à \emph{Change Directory}.

			Par exemple, si avez un répertoire \code{TP1} et que vous désirez vous y déplacer, tapez \code{cd TP1}.
			Puis si vous voulez revenir dans le répertoire parent, entrez \code{cd ../}.
			En effet, le répertoire \code{../} correspond au répertoire parent.
			Le répertoire \code{./} correspond au répertoire courant.
		\subsection{\code{pwd}}
			La commande \code{pwd} permet de savoir où on se trouve dans l'arborescence de répertoires.
			\code{pwd} correspond à \emph{Print Working Directory}.
			Si votre login est \code{toto} et que vous venez de démarrer un terminal, la commande \code{pwd} vous donnera \code{/home/toto}.
		\subsection{\code{mkdir}}
			\code{mkdir} Permet de créer un nouveau répertoire dans le répertoire courant (le répertoire dans lequel vous vous trouvez).
			\code{mkdir} correspond à \emph{MaKe DIRectory}.
			Par exemple, si vous désirez créer un nouveau répertoire \code{TP1}, entrez la commande \code{mkdir TP1}.
		\subsection{\code{man}}
			Cette commande vous permettra de vous renseigner sur n'importe quelle autre commande.
			Vous découvrirez également les différentes options possibles de chaque commande.
			\code{man} correspond à \emph{MANual}.

			Vérifiez et découvrez les options de la commande \code{ls} en entrant \code{man ls}.
	\section{Éditeur de texte}
		Avant de commencer à écrire vos programmes, il vous faut un éditeur de texte.
		Plusieurs sont disponibles, nous allons essayer les deux éditeurs les plus répandus sur les système de type \Nom{unix} : \code{emacs} et \code{vi}. 
		Bien que d'autres éditeurs existent, ces deux éditeurs font parti de la plupart des systèmes \Nom{Unix}.
		\code{emacs} comme \code{vim} bénéficient de la majorité des avantages nécessaires pour la programmation tels que la coloration syntaxique, l'indentation intelligente, la complétion automatique\dots{}
		\subsection{\code{emacs}}
			Pour utiliser \code{emacs}, de nombreux raccourcis claviers sont nécessaires.
			Découvrons les par un exemple.
			\begin{itemize}
				\item Créer le fichier \code{TP1-emacs.c} par la commande \code{emacs TP1-emacs.c};
				\item Ajouter un commentaire \code{/* Commentaires */} dans le fichier vide;
				\item Effectuer une recherche de texte avec \code{Ctrl-s};
				\item Remplacer \code{Commentaire} par \code{Bonjour tout le monde} en utilisant \code{Esc %};
				\item Enregistrer le fichier en pressant \code{Ctrl-x Ctrl-s};
				\item Quitter avec \code{Ctrl-x Ctrl-c}.
			\end{itemize}
			\subsubsection{Commandes utiles}
				\begin{description}
					\item[\texttt{Ctrl-x Ctrl-f} --] ouvrir un fichier dans le buffer courant;
					\item[\texttt{Ctrl-x k} --] fermer un buffer;
					\item[\texttt{Ctrl-x Ctrl-c} --] quitter;
					\item[\texttt{Ctrl-x i} --] insérer un fichier dans le buffer courant;
					\item[\texttt{Ctrl-x b} --] passer à un autre buffer;
					\item[\texttt{Ctrl-x 1} --] aucune division de la fenêtre;
					\item[\texttt{Ctrl-x 2} --] division horizontale de la fenêtre;
					\item[\texttt{Ctrl-x 3} --] division verticale de la fenêtre;
					\item[\texttt{Ctrl-x o} --] passer dans la prochaine sous-fenêtre;
					\item[\texttt{Ctrl-h i} --] affichage des pages \code{info} (version améliorer de \code{man}).
				\end{description}
		\subsection{\code{vim}}
			\code{vim} est un éditeur de texte dit \emph{modal}.
			Ce type d'éditeur de texte permet d'entrer des commandes dans le mode \emph{normal}, d'écrire dans le fichier dans le mode \emph{insertion} ou \emph{remplacement}.b
			Nous allons effectuer le même exemple que précédemment.
			\begin{itemize}
				\item Créer le fichier \code{TP1-vim.c} par la commande \code{vim TP1-vim.c};
				\item Pour insérer la chaîne de commentaire, il faut entrer dans le mode \emph{insertion} en tapant \code{i};
				\item Entrer normalement la chaîne de commentaire \code{/* Commentaire */};
				\item Pour effectuer une recherche, on sort du mode \emph{insertion} avec \code{Esc};
				\item Rechercher le texte \code{Commentaire} avec \code{/Commentaire};
				\item Remplacer avec \code{:%s/Commentaire/Bonjour tout le monde/};
				\item Enregistrer et quitter avec \code{:wq}.
			\end{itemize}
			\subsubsection{Commandes utiles}
				Toutes les commandes résumées ci-dessous sont valables dans le mode \emph{normal}.
				\begin{description}
					\item[\texttt{:w} --] sauvegarder;
					\item[\texttt{:q} --] quitter;
					\item[\texttt{:q!} --] quitter même si le fichier n'a pas été sauvegarder;
					\item[\texttt{:wq} --] sauvegarder et quitter;
					\item[\texttt{:x} --] sauvegarder et quitter;
					\item[\texttt{ZZ} --] sauvegarder et quitter;
					\item[\texttt{x} --] supprimer un caractère;
					\item[\texttt{dd} --] supprimer une ligne;
					\item[\texttt{yy} --] copier une ligne;
					\item[\texttt{p} --] coller une ligne;
					\item[\texttt{/} --] rechercher.
				\end{description}
	\section{Premier programme}
		Nous allons maintenant commencer la programmation par un petit exemple de programme simple.
		Créer un répertoire \code{TP1} puis placez vous dans ce nouveau répertoire.
		Puis éditez avec votre éditeur de texte un fichier \code{TP1-toto.c} en remplaçant \code{toto} par votre propre login.
		\subsection{Hello World !}
			Pourquoi ce titre ?
			Commençons par un petit peu de culture informatique.
			\emph{Hello World !} est l'expression utilisée en informatique pour tout programme élementaire.
			C'est l'exemple de base.
			Écrivez le programme \CodRef{cod-HelloWorld}.
			\begin{Code}{Hello World!}{cod-HelloWorld}
# include <stdio.h>

int main( int argc, char ** argv )
{
	printf( "Hello World!" );
	return 0;	
}
			\end{Code}

			Recopiez les lignes de ce programme dans le fichier précédemment ouvert.
			On expliquera les différentes lignes de ce programme ultérieurement.
			Pour l'instant, nous allons compiler puis exécuter ce programme.
		\subsection{Compilation}
			La compilation est une étape nécessaire pour la création d'un programme.
			Cette phase consiste à traduire le fichier précédemment écrit dans un langage compréhensible par un ordinateur.
			Ce langage se nomme l'assembleur.
			Le compilateur par excellence sur un système \Nom{unix} est \code{gcc}.

			Nous allons donc compiler avec la commande
			\begin{Code*}
$~$ gcc -o tp1 TP1.c -Wall -ansi
			\end{Code*}
			L'option \code{-o tp1} permet de donner le nom au programme qui va être créé.
			L'option \code{-Wall} permet d'informer plus précisément sur des erreurs et d'en prévoir d'autres.
			L'option \code{-ansi} permet de vérifier que le code source respecte les normes de programmation.

			À présent, nous allons exécuter le programme en entrant la commande
			\begin{Code*}
$~$ ./tp1
			\end{Code*}
			Vous devriez voir quelque chose s'afficher.
			Cependant, la mise en page du message laisse à désirer; il serait intéressant de pouvoir passer à la ligne à la fin du message.
			Pour effectuer un passage à la ligne dans une chaîne de caractère, on ajoute le caractère \code{'\n'} qui représente une fin de ligne.
			On modifie donc la ligne~5 du programme par
			\begin{Code*}[numbers=left,firstnumber=5]
	printf( "Hello World!\n" );
			\end{Code*}
			Maintenant, reprenons la compilation et l'exécution
			\begin{Code*}
$~$ gcc -o tp1 TP1.c -Wall -ansi
$~$ ./td1
Hello World!
$~$
			\end{Code*}
		\subsection{La fonction \code{printf}}
			La fonction \code{printf} est une fonction offerte par défaut dans tous les environnements de développement.
			Elle est très utile pour afficher différents types de données.
			Par exemple, elle va nous permettre d'afficher des nombres.

			Imaginons que nous désirons afficher la valeur \code{1}.
			Bien sûr, on peut se contenter d'écrire \code{printf("1")}.
			Mais si \code{1} est le résultat d'un calcul, comment fait-on ?

			Pour cela, on va utiliser les \emph{caractères de contrôle} qui s'insèrent dans la chaîne de caractères.
			Un caractère de contrôle commence par le caractère \code{'%'}.
			À la suite de ce caractère se trouve la nature de la valeur qui sera affichée :
			\begin{itemize}
				\item \code{d} pour afficher un entier;
				\item \code{f} pour afficher un réel;
				\item \code{c} pour afficher un caractère;
				\item \code{s} pour afficher une chaîne de caractères.
			\end{itemize}
			Pour afficher un entier, on écrira donc \code{printf( "%d", 1 );}.
			Dans notre cas, nous allons afficher la date.
			Modifions à nouveau la ligne~5 par
			\begin{Code*}[numbers=left,firstnumber=5]
	printf( "La date du jour est %d-%d-%d\n", 3, 9, 2008 );
			\end{Code*}

			On peut également afficher un réel avec \code{"%f"}.
			Il est possible de paramétrer le nombre de chiffre à afficher après la virgule en écrivant \code{"%.3f"} (pour 3 chiffres après la virgule).
			Modifier le programme pour afficher le chiffre \nombre{1,23456789} avec 4~chiffres après la virgule.

			D'autres possibilités avec la fonction \code{printf} existent.
			Je vous invite à regarder sur Internet la documentation de la fonction \code{printf} et répondre aux questions suivantes : comment écrit-on un entier en hexadécimal ?
			Et en octal ?
			La recherche sur Internet fait partie intégrante du travail d'un programmeur donc n'hésitez pas à aller fouiller sur la toile pour trouver des solutions à vos problèmes de programmation.

\end{document}
