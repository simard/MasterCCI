\documentclass[a4paper]{article}
	\usepackage{Lapin}
	\newenvironment{Correction}{\par\tiny\blue\rule[1ex]{\textwidth}{1pt}\par\normalsize\textbf{\sffamily{}Correction de l'exercice~\theExo{} -- }}{\par\tiny\blue\rule[1ex]{\textwidth}{1pt}\par}
	\newtheorem{Exo}{{\sffamily{Exercice}}}
	\title{Type de données}
	\author{Philippe \Nom{Rinaudo}\and{}Jean \Nom{Simard}}
	\date{\Date[l]{06}{10}{2009}}
\begin{document}
	\maketitle
	\section{Quelques conseils pour bien programmer}
		Nous allons commencer par donner quelques bonnes habitudes à prendre pour avoir une programmation lisible et simplifiée.
		Une des premières bonne manière de programmer est d'utiliser un éditeur de texte doté de la coloration syntaxique.
		Une chance, la plupart sont déjà pourvu de cette fonctionnalité et vous n'avez rien à faire.
		Mais parlons des habitudes que vous aurez à prendre.
		\subsection{L'indentation}
			Qu'est-ce que l'indentation ?
			Vous avez eu l'occasion de découvrir que le langage \bsc{c} s'écrit avec des blocs délimités par \code+{+ et \code+}+.
			D'ailleurs, il en de même pour la plupart des langages de programmation.
			Il est donc intéressant de pouvoir apercevoir au premier coup d'œil ces blocs.
			L'indentation va nous permettre de mettre en page le code source de façon à faire ressortir ces blocs.

			Un exemple est plus parlant alors donnons un exemple.
			Nous allons reprendre le programme \emph{Hello World} vu au TP précédent.
			Voici la version non-indentée de ce programme.
			\begin{Code}{\emph{Hello World} non-indenté}{cod-HelloWorldNonIndente}
# include <stdio.h>
int main( int argc, char ** argv )
{
printf( "Hello World!\n" );
return 0;
}
			\end{Code}
			et voici la version indentée de ce programme.
			\begin{Code}{\emph{Hello World} indenté}{cod-HelloWorldIndente}
# include <stdio.h>
int main( int argc, char ** argv )
{
	printf( "Hello World!\n" );
	return 0;
}
			\end{Code}

			Bien sûr, la différence sur un petit programme comme celui-ci n'est pas énorme mais vous constaterez rapidement que sur des programmes plus complexes avec plusieurs blocs imbriqués, la lisibilité d'un code source indenté est bien meilleure.
		\subsection{Le nom des variables}
			Aujourd'hui, nous allons aborder les variables.
			Les variables d'un programme sont déclarées au début d'un bloc ou au début du programme.
			Ensuite, elles peuvent apparaître à n'importe quel endroit dans le programme.
			Il est donc très important de pouvoir savoir à n'importe quel endroit à quoi correspond une variable.
			C'est pour cette raison qu'il est nécessaire de lui donner un nom explicite dès le début.

			De nouveau, prenons un exemple.
			Considérons les trois variables de \emph{jour}, \emph{mois} et \emph{année} pour définir une date.
			Nous pouvons donc donner comme nom de variable \code{Jour}, \code{Mois} et \code{Annee} au trois variables.
			\textbf{Attention, n'utilisez pas d'accents dans un programme en \bsc{c}.}
			Il faut absolument éviter de donner des noms tels que \code{A}, \code{B} et \code{C} ou encore \code{tata}, \code{titi} et \code{toto}.

			Allons un peu plus loin.
			Vous avez pu constater que les variables en \bsc{c} sont typées.
			Pourquoi ne pas indiquer également le type de la variable dans son nom.
			Dans notre précédent exemple, \code{Jour}, \code{Mois} et \code{Annee} sont trois entiers donc de type \code{int}.
			Nous pourrions donc utiliser par exemple la première lettre pour indiquer le type comme par exemple \code{iJour}, \code{iMois} et \code{iAnnee}.

			Ce que je donne comme exemple ci-dessus ne sont que des propositions.
			À vous de choisir votre méthode pour choisir les noms de variables.
		\subsection{L'espacement}
			Il est également conseillé d'écrire de façon aérée.
			Ne pas être fainéant sur la touche espace donne également une meilleure lisibilité à votre code source.
			Là aussi, à vous de choisir une bonne habitude et de vous y tenir.
			L'essentiel est que vous soyez cohérent tout au long du fichier.

			Par exemple, pour l'initialisation d'une variable
			\begin{Code*}
float fMoyenne = fSomme / 10.0f;
			\end{Code*}
			est plus lisible que
			\begin{Code*}
float fMoyenne=fSomme/10.0f;
			\end{Code*}
		\subsection{Exemple / Contre-exemple}
			Je vais donner ici in exemple de code source où j'ai appliqué les règles énoncées ci-dessus et un autre où j'ai exagéré les mauvaises habitudes de programmation.
			\begin{Code}{Code avec les règles}{cod-CodeAvecLesRegles}
# include <stdio.h>

int main( int argc, char ** argv )
{
	int iJour = 7, iMois = 10, iAnnee = 2008;
	printf( "La date du jour est %d-%d-%d.\n",\
		iJour, iMois, iAnnee );

	return 0;
}
			\end{Code}
			\begin{Code}{Code sans les règles}{cod-CodeSansLesRegles}
# include <stdio.h>
int main(int argc, char **argv){int i=7,j=10,k=2008;
printf("La date du jour est %d-%d-%d.\n",i,j,k);
return 0;}
			\end{Code}

			Bien que les programmes fonctionnent tous les deux et qu'ils donnent le même résultat, il sera plus facile de relire le premier pour le modifier ultérieurement.

	\section{Les variables et leur type}
		Nous allons aujourd'hui nous familiariser avec le concept de variable qui est un concept très important de la programmation.
		Une variable est un espace de la mémoire dans lequel on souhaite sauvegarder une valeur d'un type particulier.
		Commençons par voir comment déclarer une telle variable.
		\subsection{Déclaration d'une variable}
			Afin qu'un programme \bsc{c} puisse accéder à une variable, il faut au préalable lui déclarer cette variable.
			On va donc signaler au compilateur qu'une variable existe en précisant :
			\begin{itemize}
				\item son nom qu'on appelle \emph{identificateur}\footnote{Un \emph{identificateur} peut être constitué de lettres, de chiffres ou du caractère \code{'_'}.
			Cependant, le premier caractère de la variable ne doit pas être un chiffre.};
				\item son type qui décrit la nature de la variable (\code{int}, \code{char}, \code{float}\dots{});
				\item sa valeur qui est facultative au moment de la déclaration\footnote{Bien que la valeur d'une variable soit facultative au moment de la déclaration, il est fortement conseillé de lui indiquer une valeur dès la déclaration.
				Cela permet d'éviter d'éventuelles erreurs dans la suite du programme.
				On parle alors d'initialisation de la variable.
				Cette pratique fait partie des bonnes habitudes de programmation.}.
			\end{itemize}

			Pour déclarer une variable, on commence par indiquer le type, puis on donne le nom et enfin, on l'initialise.
			Par exemple, pour déclarer une variable \code{iJour} de type entier initialisé avec la valeur \code{7}, on écrira
			\begin{Code*}
int iJour = 7;
			\end{Code*}

			De plus, dans le précédent TP, nous avons vu comment utiliser la fonction \code{printf} afin qu'elle affiche des valeurs.
			En l'occurence, pour afficher un entier, on avait utilisé le caractère de contrôle \code{%d}.
			\begin{Exo}
				Recopiez le \CodRef{cod-CodeAvecLesRegles} et compilez-le.
			\end{Exo}
			\begin{Correction}
				L'exécution du programme donne le résultat suivant.
				\begin{Code*}
La date du jour est 6-10-2009.
				\end{Code*}
			\end{Correction}

			Dans le programme que vous venez de compiler, à chaque endroit où apparaît le nom de la variable \code{iJour}, le compilateur comprend qu'il faut qu'il utilise la valeur qui correspond à cette variable, c'est-à-dire \code{7}.
			Il opère de la même façon pour les variables \code{iMois} et \code{iAnnee}.

			On peut également modifier une variable au cours du programme.
			Ceci peut être fait en écrivant
			\begin{Code*}
iJour = 2;
			\end{Code*}
			Par exemple, utilisons un programme en \bsc{c} en guise de calculatrice.
			\begin{Code}{Calculatrice simple}{cod-CalculatriceSimple}
# include <stdio.h>

int main( int argc, char ** argv )
{
	/* On commence par initialiser la variable a zero */
	int iResultat = 0;
	/* On effectue un calcul */
	iResultat = ( 87 + 113 ) * 2;
	/* Puis on affiche le resultat */
	printf( "( 87 + 113 ) * 2 = %d\n", iResultat );

	return 0;
}
			\end{Code}

			\begin{Exo}
				Recopiez le \CodRef{cod-CalculatriceSimple} et compilez-le.
			\end{Exo}
			\begin{Correction}
				L'exécution du programme donne le résultat suivant.
				\begin{Code*}
( 87 + 113 ) * 2 = 400
				\end{Code*}
			\end{Correction}

		\subsection{Portée d'une variable}
			Un programme \bsc{c} classique est composé de plusieurs blocs.
			Si une variable est déclarée dans un bloc, elle existera dans ce bloc et seulement dans ce bloc.
			On dira alors que cette variable est une variable \emph{locale}.

			Par opposition, on dira qu'une variable déclarée en dehors de tout bloc sera une variable \emph{globale}.
			\begin{Code}{Variable globale vs Variable locale}{cod-VariableGlobalevsVariableLocale}
# include <stdio.h>
int iVarGlobale = 0;

int main( int argc, char ** argv )
{
	int iVarLocale = 1;
	/* Nouveau bloc */
	{
		int iAutreVarLocale = 2;
	}
	printf( "iVarGlobale = %d\n", iVarGlobale );
	printf( "iVarLocale = %d\n", iVarLocale );
	printf( "iAutreVarLocale = %d\n", iAutreVarLocale );

	return 0;
}
			\end{Code}
			\begin{Exo}
				Recopiez le \CodRef{cod-VariableGlobalevsVariableLocale} et compilez-le.
			\end{Exo}
			\begin{Correction}
				La compilation de ce programme nous donne les erreurs suivantes.
				\begin{Code*}
GlobalvsLocal.c: In function 'main':
GlobalvsLocal.c:10: warning: unused variable 'iAutreVarLocale'
GlobalvsLocal.c:14: error: 'iAutreVarLocale' undeclared (first use in this function)
GlobalvsLocal.c:14: error: (Each undeclared identifier is reported only once
GlobalvsLocal.c:14: error: for each function it appears in.)
				\end{Code*}
			\end{Correction}
			\begin{Exo}
				Pourquoi la compilation donne-t-elle une erreur ? Corrigez le programme pour qu'il compile.
			\end{Exo}
			\begin{Correction}
				La variable \code{iAutreVarLocale} étant déclarée dans un bloc indépendant, elle n'existe que dans ce bloc donc elle n'est pas accessible en dehors de ce bloc.
				Voici une correction possible de ce code source.
				\FichierCode{GlobalvsLocal.c}{cod-GlobalVsLocalC}
			\end{Correction}

		\subsection{Entrées / Sorties}
			Jusqu'à présent, nous avons vu comment effectuer un affichage sur l'écran à l'aide de la fonction \code{printf}.
			On appelle cet affichage une \emph{sortie} du programme car le programme effectue une opération qui est visible par l'utilisateur.
			Maintenant que nous avons vu comment le programme peut communiquer avec l'utilisateur, voyons comment faire pour que l'utilisateur communique avec l'ordinateur.

			Commençons donc par présenter la fonction \code{getchar} qui permet de récupérer un caractère entré par l'utilisateur.
			Cette fonction renvoie un entier qui correspond à la valeur \bsc{ascii} du caractère.
			\begin{Code}{Fonction \code{getchar}}{cod-FonctionGetchar}
# include <stdio.h>

int main( int argc, char ** argv )
{
	if( getchar() == 'Y' )
	{
		printf("Yes\n") ;
	} else {
		printf("No\n") ;
	}

	return 0;
}
			\end{Code}
			\begin{Exo}
				Recopiez le \CodRef{cod-FonctionGetchar} et compilez-le.
			\end{Exo}
			\begin{Correction}
				Le programme écrit \code{Yes} lorsque l'utilisateur valide la touche \code{'Y'} et écrit \code{No} pour toutes les autres touches.
			\end{Correction}

			Pour récupérer des expressions plus complexes, on peut utiliser la fonction \code{scanf}.
			Cette fonction permet de lire plusieurs champs grâce à des caractères de contrôle qui fonctionnent de la même façon que la fonction \code{printf}
			\begin{Code}{Fonction \code{scanf}}{cod-FonctionScanf}
# include <stdio.h>

int main( int argc, char ** argv )
{
	int iJour = 0, iMois = 0, iAnnee = 0;
	printf( "Entrer une date (jj/mm/aaaa)\n" );
	scanf( "%d/%d/%d", &iJour, &iMois, &iAnnee );
	printf( "La date est %d-%d-%d\n",\
		iJour, iMois, iAnnee );

	return 0;
}
			\end{Code}
			\begin{Exo}
				Recopiez le \CodRef{cod-FonctionScanf} et compilez-le.
			\end{Exo}
			\begin{Correction}
				L'exécution du programme donne le résultat suivant.
				\begin{Code*}
Entrer une date (jj/mm/aaaa)
06/10/2009
La date est 6-10-2009
				\end{Code*}
				À la deuxième ligne, c'est l'utilisateur qui a entré les valeurs.
			\end{Correction}

	\section{Nombres entiers et réels}
		Nous avons eu l'occasion de voir comment déclarer une variable.
		Nous allons à présent nous intéresser aux variables de type numérique.
		\subsection{Le type \code{int}}
			Le type \code{int} permet de préciser les variables pour les nombres entiers.
			La plupart du temps, les variables de type \code{int} sont codées sur 32~bits soit 4~octets mais cela dépend de l'ordinateur utilisé.
			Les valeurs de types \code{int} sont signées donc la plage de valeur définie est comprise entre $-2^{31}$ et $2^{31}-1$ soit l'intervalle $[-\nombre{2147483648},\nombre{2147483647}]$.

			On peut décliner le type \code{int} avec plusieurs précisions en utilisant des informations supplémentaires.
			Par exemple, si on veut utiliser un entier non-signé, on utilisera \code{unsigned int}.
			Ou si la taille d'un entier simple ne suffit pas, on peut utiliser \code{long int}.
			Pour un petit entier, on utilise \code{short int}.
			À noter que le type \code{char} est aussi considéré comme un type entier et sauvegarde la valeur entière correspondant au code \bsc{ascii}.

			Pour connaître le nombre d'octets utilisés pour les différents types présenté ci-dessus, on peut utiliser la fonction \code{sizeof}.
			Cette fonction est une fonction native du langage \bsc{c} contrairement à \code{printf}.
			Elle peut être utilisée pour calculer le nombre d'octet de différents type de variable.
			Par exemple, pour déduire la taille de 4~octets pour un entier, on utilise
			\begin{Code*}
printf( "Taille de int = %d\n", sizeof( int ) );
			\end{Code*}
			
			\begin{Exo}
				Écrivez un programme qui donne la taille des différents types entiers présentés ci-dessus.
			\end{Exo}
			\begin{Correction}
				Ici, aucun besoin de déclarer une quelconque variable.
				Cependant, on peut toutefois souligner que la fonction \code{sizeof} peut aussi prendre en paramètre le nom d'une variable.
				Dans ce cas, il en déduit le type de la variable et calcul la taille en octets, même pour des types structurés.
				\FichierCode{TailleTypes.c}{cod-TailleTypesC}
			\end{Correction}
			\begin{Exo}
				En déduire les plages de nombres pour chacun des types.
			\end{Exo}
			\begin{Correction}
				Pour connaître les plages de nombres disponible pour un type entier, il faut utiliser le nombre d'octets $n$ pour définir l'intervalle $[-2^{n-1},2^{n-1}-1]$.
				Pour le cas particulier des entiers non-signés, l'intervalle sera alors $[0,2^n-1]$.

				Par exemple, pour le type \code{unsigned char} qui est codé sur \nombre{1}~octet, on en déduit un intervalle $[0,255]$.
			\end{Correction}

		\subsection{Les types \code{float} et \code{double}}
			Pour les variables de type réel, on utilise \code{float} ou \code{double}.
			\begin{Exo}
				À partir du programme précédent, donnez la taille de ces deux types.
			\end{Exo}
			\begin{Correction}
				\FichierCode{TailleReels.c}{cod-TailleReelsC}
			\end{Correction}
			Bien que vous veniez de définir le nombre de bits utilisés pour ces deux types, il vous est impossible d'en déduire une plage de nombre.
			En effet, les nombres sont sauvegardés sous le principe des \emph{nombres à virgule flottante}.
			Cela signifie que le nombre de bits utilisés pour la partie entière est variable.
			Ce qui n'est pas utilisé pour la partie entière est reporté pour la partie décimale et vice versa.

			L'utilisation des nombres à virgule flottante est plus complexe et peut souffrir d'approximations provoquant des résultats inexacts.
			\begin{Code}{Imprécision sur les \code{float}}{cod-ImprecisionSurLesFloat}
# include <stdio.h>

int main( int argc, char ** argv )
{
	printf( "Un surprenant reel : %.17f\n", 1.1 + 2.7 );

	return 0;
}
			\end{Code}
			\begin{Exo}
				Recopiez le \CodRef{cod-ImprecisionSurLesFloat} et compilez-le.
			\end{Exo}
			\begin{Correction}
				L'exécution du programme donne le résultat suivant.
				\begin{Code*}
Un surprenant reel : 3.80000000000000027
				\end{Code*}
				Avec \code{%.17f}, la fonction \code{printf} affiche une précision de 17~chiffres après la virgule.
				On remarque alors que l'addition de \nombre{1.1} et \nombre{2.7} ne donne pas exactement \nombre{3.8} mais n'est qu'une approximation du résultat.
				Mais peut-être la représentation des nombres \nombre{1.1} et \nombre{2.7} dans l'ordinateur n'est pas exactement celle qu'on imagine également !
			\end{Correction}

		\subsection{Opérateurs}
			Les opérateurs en \Nom{c} sont nombreux et variés (voir \TabRef{tab-ListeDOperateurs}).
			Ils sont valables pour certains types de données et peuvent être incompatible entre types de données mixtes comme par exemple entre un \code{int} et un \code{float}.
			Le plus souvent, il n'y aura pas d'erreur de compilation car une conversion sera effectuée mais \textbf{attention, les résultats peuvent ne pas être ceux que vous attendez}.
			Donc veillez toujours à combiner des types compatibles et si le besoin s'en fait sentir, à bien spécifier explicitement des conversions.
			Ne laissez pas le compilateur le faire à votre place, il faut garder le contrôle sur ce que vous faites.
			\begin{Table}{Listes d'opérateurs}{tab-ListeDOperateurs}{|l|c|r@{\hspace{0.5cm}$\rightarrow$\hspace{0.5cm}}l|}
				\hline
				\multicolumn{1}{|c|}{\textbf{Nom}} & \textbf{Notation} & \multicolumn{2}{|c|}{\textbf{Exemple}} \\
				\hline
				\hline
				\multicolumn{4}{|c|}{Opérateurs arithmétiques}\\
				\hline
				Addition & \verb$x + y$ & \verb$8 + 6$ & \verb$14$ \\
				\hline
				Soustraction & \verb$x - y$ & \verb$8 - 6$ & \verb$2$ \\
				\hline
				Multiplication & \verb$x * y$ & \verb$8 * 6$ & \verb$48$ \\
				\hline
				Division entière & \verb$x / y$ & \verb$8 / 6$ & \verb$1$ \\
				\hline
				Division réelle & \verb$x / y$ & \verb$8.0 / 6.0$ & \verb$1.333$ \\
				\hline
				Modulo (reste euclidien) & \verb$x % y$ & \verb$8 % 6$ & \verb$2$ \\
				\hline
				\hline
				\multicolumn{4}{|c|}{Opérateurs de comparaison}\\
				\hline
				Égalité & \verb$x == y$ & \verb$8 == 6$ & \verb$0$ \\
				\hline
				Inégalité & \verb$x != y$ & \verb$8 != 6$ & \verb$1$ \\
				\hline
				Supérieur & \verb$x > y$ & \verb$8 > 6$ & \verb$1$ \\
				\hline
				Inférieur & \verb$x < y$ & \verb$8 < 6$ & \verb$0$ \\
				\hline
				Supérieur ou égal & \verb$x >= y$ & \verb$8 >= 6$ & \verb$1$ \\
				\hline
				Inférieur ou égal & \verb$x <= y$ & \verb$8 <= 6$ & \verb$0$ \\
				\hline
				\hline
				\multicolumn{4}{|c|}{Opérateurs logiques}\\
				\hline
				\bsc{et} & \verb$x && y$ & \verb$8 && 6$ & \verb$1$ \\
				\hline
				\bsc{ou} & \verb$x || y$ & \verb$8 || 6$ & \verb$1$ \\
				\hline
				\bsc{non} & \verb$!x$ & \verb$!8$ & \verb$0$ \\
				\hline
				\hline
				\multicolumn{4}{|c|}{Opérateurs bit à bit}\\
				\hline
				\bsc{et} & \verb$x & y$ & \verb$1000 & 1010$ & \verb$1000$ \\
				\hline
				\bsc{ou} inclusif & \verb$x | y$ & \verb$1000 | 1010$ & \verb$1010$ \\
				\hline
				\bsc{ou} exclusif & \verb$x ^ y$ & \verb$1000 ^ 1010$ & \verb$0010$ \\
				\hline
				Décalage à droite & \verb$x >> y$ & \verb$1000 >> 2$ & \verb$0010$ \\
				\hline
				Décalage à gauche & \verb$x << y$ & \verb$0010 << 1$ & \verb$0100$ \\
				\hline
				\hline
				\multicolumn{4}{|c|}{Opérateurs particuliers}\\
				\hline
				Incrémentation & \verb$x++$ & \verb$x++$ & \verb$x + 1$ \\
				\hline
				Décrémentation & \verb$x--$ & \verb$x--$ & \verb$x - 1$ \\
				\hline
			\end{Table}

	\subsection{La conversion de types}
		Habituellement, les entiers et les réels sont tous les deux des nombres et une opération entre un réel et un entier est correcte dans le domaine des mathématiques.
		Cependant, étant donné que dans le langage \bsc{c}, les nombres entiers ne sont pas codés de la même façon que les nombres réels, la conversion en devient plus complexe.
		Heureusement, le langage \bsc{c} nous offre la possibilité de faire les transformations mais en contre-partie, il nous faudra lui indiquer précisément où doivent s'effectuer ces conversions.

		Par exemple, si la variable \code{iEntier} est de type \code{int} et que la variable \code{fReel} est de type \code{float}, on ne pourra pas additionner ces deux variables puisqu'elle ne sont pas de même type.
		On convertira donc par exemple la variable \code{iEntier} en \code{float} pour ls besoins de cette addition.
		On écrira alors le nouveau type juste devant le nom de la variable en utilisant des parenthèses.
		\begin{Code*}
fResultat = (float)iEntier + fReel;
		\end{Code*}

		\begin{Code}{Code sans conversion}{cod-CodeSansConversion}
# include <stdio.h>

int main( int argc, char ** argv )
{
	printf( "Le resultat est %f\n", 3 / 2 );

	return 0;
}
		\end{Code}
		\begin{Exo}
			Recopiez le \CodRef{cod-CodeSansConversion}, compilez-le puis executez-le.
		\end{Exo}
			\begin{Correction}
				L'exécution du programme donne le résultat suivant.
				\begin{Code*}
					Le resultat est -1.992378
				\end{Code*}
				Puis une deuxième exécution du programme donne le résultat suivant.
				\begin{Code*}
					Le resultat est -1.625099
				\end{Code*}
				Puis une troisième donne encore un résultat différent.
				Ce n'est pas du tout le résultat attendu.
				Cependant, si vous avez utilisé l'option de compilation \code{-Wall}, vous avez pu constater que le compilateur vous avertissait d'une possible erreur.
				\begin{Code*}
Conversion.c: In function 'main':
Conversion.c:5: warning: format '%f' expects type 'double', but argument 2 has type 'int'
				\end{Code*}
			\end{Correction}
		\begin{Exo}
			Le résultat n'est pas celui attendu ! Corrigez le code pour obtenir le résultat \nombre{1,5} (deux solutions possibles).
		\end{Exo}
		\begin{Correction}
			La première solution consiste à faire une conversion explicite.
			À noter que dans le jargon informatique, on dit qu'on \emph{force} la valeur ou quelque fois, qu'on \emph{caste} la valeur.
			Cependant, ce dernier terme est un anglicisme et n'est donc par correct en français.
			Il provient du terme anglais \emph{casting}.
			\FichierCode{Conversion1.c}{cod-Conversion1C}
			Mais on peut également modifier les nombres directement pour les exprimer sous forme de nombre réel.
			\FichierCode{Conversion2.c}{cod-Conversion2C}
		\end{Correction}

		Les conversions ne sont pas toujours des opérations obligatoires.
		En effet, certaines conversions peuvent être effectuées de façon implicite.
		Cependant, il est fortement déconseillée de faire appel à ce genre de pratique.
		C'est une nouvelle bonne habitude de programmation d'effectuer des conversions explicites.

	\section{Les structures}
		Les types \code{int} et \code{float} sont des types de base.
		Nous verrons qu'il en existe d'autres (comme \code{double} ou \code{char} par exemple).
		Mais parfois, les types de base ne suffisent pas et des types plus évolués sont nécessaires.
		Ils peuvent être définis par l'utilisateur et sont appelés \emph{types structurés}.
		Il existe plusieurs types structurés différents comme les tableaux ou les unions mais ici, nous verrons les structures.
		\subsection{Définition}
			Un type structuré est un type constitué de plusieurs type de base.
			C'est donc une sorte de liste de variables regroupées dans un groupe.
			Si par exemple, nous voulons définir une structure pour un étudiant, nous pourrions créer la structure suivante
			\begin{Code*}
struct sEtudiant
{
	int iNumEtudiant;
	char * cPrenom;
	char * cNom;
};
			\end{Code*}

			\begin{Exo}
				Créez une structure permettant de représenter un point dans le plan avec les champs \code{fAbscisse} et \code{fOrdonnee}.
				Les champs seront de type \code{float} et la structure se nommera \code{sPoint}.
			\end{Exo}
			\begin{Correction}
				\begin{Code}{Structure \code{sPoint}}{cod-StructureSPoint}
struct sPoint
{
	float fAbscisse;
	float fOrdonnee;
};
				\end{Code}
			\end{Correction}

		\subsection{Utilisation}
			Une fois la structure définie, il faut alors déclarer une variable du type de la structure.
			Puis on pourra accéder aux champs de cette variable grâce au caractère \code{'.'}.
			Prenons un exemple où nous écrivons la valeur d'un champ de la structure dans une variable basique.
			\begin{Code*}
int iUnNumero = 0;
struct sEtudiant sQuelquUn = { 123456789, "Jean", "SIMARD" };
iUnNumero = sQuelquUn.iNumEtudiant;
			\end{Code*}
			Dans cet exemple, on peut aussi remarquer la manière d'initialiser une structure.
			Les champs sont initialisés par des accolades où chaque champ est séparé par une virgule.

			\begin{Exo}
				Récupérez dans deux variables \code{fX} et \code{fY} les valeurs d'une structure de type \code{sPoint} initialisée par $(0,0)$ (l'origine du repère).
			\end{Exo}
			\begin{Correction}
				\FichierCode{Structure.c}{cod-StructureC}
			\end{Correction}
\end{document}
