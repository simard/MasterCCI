\documentclass[a4paper]{article}
	\usepackage{Lapin}
	\newenvironment{Correction}{\par\tiny\blue\rule[1ex]{\textwidth}{1pt}\par\normalsize\textbf{\sffamily{}Correction de l'exercice~\theExo{} -- }}{\par\tiny\blue\rule[1ex]{\textwidth}{1pt}\par}
	\newtheorem{Exo}{{\sffamily{Exercice}}}
	\title{Branchements conditionnels}
	\author{Philippe \Nom{Rinaudo}\and{}Jean \Nom{Simard}}
	\date{\Date[l]{08}{10}{2009}}
\begin{document}
	\maketitle
	\section{Le branchement \code{if}}
		Le branchement \code{if} est appelé \emph{branchement conditionnel}.
		C'est le branchement conditionnel de base.
		Il permet d'effectuer une opération seulement si une condition est respectée.
		Si cette condition n'est pas respectée, il est possible d'effectuer une autre opération.

		Commençons par la clause \code{if}.
		On rédigera
		\begin{Code*}
if( <condition> )
{
	<operation 1>;
}
		\end{Code*}

		Dans le cas où on désire qu'une autre opération soit effectuée si la condition n'est pas respectée, on utilisera la clause \code{else}.
		\begin{Code*}
if( <condition> )
{
	<operation 1>;
}
else
{
	<operation 2>;
}
		\end{Code*}

		\code{<condition>} est une expression booléenne qui retourne soit \code{1}, soit \code{0}.
		La condition est considérée comme vraie si l'expression booléenne retourne \code{1}.
		On en déduit que la clause \code{else} (\code{<operation 2>}) est exécutée si \code{<condition>} retourne \code{0}.

		\begin{Exo}
			Écrire un programme qui propose d'entrer son âge et qui affiche, selon l'âge, si la personne est majeure ou non.
			\textbf{Rappel} : L'âge de la majorité en \bsc{France} est 18~ans.
		\end{Exo}
		\begin{Correction}
			Pour la correction de cet exercice, on peut utiliser quelques notions vues au TP précédent.
			Dans ce cas, un âge ne peut pas être un nombre négatif donc on va utiliser le mot-clef \code{unsigned}.
			\FichierCode{MajeurMineur.c}{cod-MajeurMineur}
		\end{Correction}
		\begin{Exo}
			Écrire un programme qui demande d'entrer un nombre entre $0$ et $10$ correspondant à une note.
			Si le nombre est inférieur à $5$, afficher «~\emph{Vous n'avez pas la moyenne}~», si le nombre est supérieur ou égal à la moyenne, afficher «~\emph{Vous avez la moyenne}~».
			Vérifiez également que le nombre entré est compris entre $0$ et $10$, sinon, afficher «~\emph{Note incorrecte}~».
		\end{Exo}
		\begin{Correction}
			Ici nous allons utiliser pour la première fois des opérateurs booléens (\code{&&}).
			\FichierCode{Note.c}{cod-NoteC}
		\end{Correction}

	\section{Le branchement \code{switch}}
		Le branchement conditionnel \code{switch} permet d'effectuer différentes opérations en fonction de la valeur d'une variable.
		Contrairement au branchement \code{if}, la condition testée n'est pas booléenne mais numérique.
		On peut donc accéder à de multiples branches.
		\begin{Code*}
switch( <expression> )
{
	case <valeur 1> :
		<operation 1>;
	case <valeur 2> :
		<operation 2>;
	case <valeur 3> :
		<operation 3>;
	...
	default :
		<operation n>;

}
		\end{Code*}

		Si \code{<expression>} est égal à \code{<valeur 2>}, alors c'est \code{<operation 2>} qui est exécuté.
		Si \code{<expression>} ne correspond à aucune valeur, alors \code{<operation n>} est exécuté (clause \code{default}).
		Cependant, lorsqu'une branche est validée, toutes les opérations qui suivent sont exécutées.
		Par exemple, si \code{<expression>} est égal à \code{<valeur 1>}, alors \code{<operation 1>} est executé mais aussi \code{<operation 2>}, \code{<operation 3>}\dots{} jusqu'à \code{<operation n>}.
		Bien souvent, un tel fonctionnement n'est pas très utile.
		Mais le langage \bsc{c} propose le mot-clef \code{break} pour résoudre ce problème.
		Lorsque le programme rencontre le mot \code{break} au sein du branchement conditionnel \code{switch}, il sort directement de ce branchement conditionnel.
		On peut donc écrire à nouveau un modèle plus complet du branchement conditionnel \code{switch}.
		\begin{Code*}
switch( <expression> )
{
	case <valeur 1> :
		<operation 1>;
		break;
	case <valeur 2> :
		<operation 2>;
		break;
	case <valeur 3> :
		<operation 3>;
		break;
	...
	default :
		<operation n>;
		break;

}
		\end{Code*}

		\begin{Exo}\label{exo-MenuAChoix}
			Écrire un programme qui lit un caractère en entrée et :
			\begin{itemize}
				\item propose d'entrer son âge et affiche «~\emph{Majeur}~» ou «~\emph{Mineur}~» (suivant l'âge entré) lorsque le caractère \code{'a'} est entré;
				\item affiche «~\emph{Au revoir}~» lorsque le caractère \code{'q'} est entré;
				\item affiche «~\emph{Choix non-autorise}~» lorsqu'un autre caractère est entré.
			\end{itemize}
		\end{Exo}
		\begin{Correction}
			Pour le choix du caractère \code{'a'}, on va réutiliser le code source de l'exercice précédent \CodRef[v]{cod-MajeurMineur}.
			\FichierCode{SwitchMenu.c}{cod-SwitchMenu}
		\end{Correction}
		\begin{Exo}
			Reprendre l'exercice~\vref{exo-MenuAChoix} et le modifier pour que les lettres prises en compte soient les minuscules et les majuscules.
		\end{Exo}
		\begin{Correction}
			Pour ajouter les majuscules dans le programme, on va reprendre le programme précédent \CodRef[v]{cod-SwitchMenu}.
			On va simplement rajouter le cas de la majuscule à côté de la lettre en minuscule.
			\begin{Code*}
	...
	switch( getchar() )
	{
		case 'a' :
		case 'A' :
		{
			...
		}
		case 'q' :
		case 'Q' :
		{
			...
		}
	}
			\end{Code*}
		\end{Correction}
	\section{Approfondissement}
		\begin{Exo}[Conversions]
			Nous allons créer un programme qui va nous permettre de convertir une température entre l'échelle de \bsc{Celsius}, l'échelle de \bsc{Fahrenheit} ou l'échelle de \bsc{Kelvin}\footnote{Pour connaître les formules de conversion, allez voir le site \url{http://fr.wikipedia.org}}.

			Le programme doit commencer par proposer une échelle pour la température à convertir.
			Ensuite, l'utilisateur doit pouvoir entrer la valeur de la température à convertir.
			Enfin, le programme doit proposer l'échelle dans laquelle la température doit être convertie.

			Voici un exemple d'exécution du programme.
			\begin{Code*}[language=]
$~$ ./conversion
### Programme de conversion ###
1 - Kelvin
2 - Celsius
3 - Fahrenheit
Choisissez l'echelle de la temperature a convertir : 1
Entrez la temperature a convertir : 273.15
1 - Celsius
2 - Fahrenheit
Choisissez l'echelle de conversion : 1

Une temperature de 273.15K correspond a 0.0C.
### Fin du programme ###
$~$
			\end{Code*}
		\end{Exo}
		\begin{Correction}
			\FichierCode{Conversion.c}{cod-ConversionC}
		\end{Correction}
		\begin{Exo}[Calculatrice]
			Nous avons vu au dernier TP un exemple de calculatrice assez basique.
			Nous allons maintenant combiner nos connaissances pour créer une calculatrice un peu plus interactive.
			L'utilisateur devra pouvoir entrer une expression du type \code{34+45} et le programme devra lui donner le résultat.
			Les opérateurs qui devront être pris en compte sont \code{'+'}, \code{'-'}, \code{'*'} et \code{'/'}.

			Voici un exemple d'exécution du programme.
			\begin{Code*}[language=]
$~$ ./calculatrice
### Calculatrice ###
Entrez un calcul (A<operateur>B)
12+25
Le resultat est 37.
### Fin du programme ###
$~$
			\end{Code*}
			Utilisez pour cela la fonction \code{scanf} et récupérez la valeur renvoyée par cette fonction afin de quitter lors d'une erreur.
		\end{Exo}
		\begin{Correction}
			Dans cet exemple, on va utiliser d'une façon plus complexe la fonction \code{scanf} pour lire et initialiser plusieurs variables en une seule instruction.
			\FichierCode{Calculatrice.c}{cod-CalculatriceC}
		\end{Correction}
\end{document}
