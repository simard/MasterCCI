\documentclass[a4paper]{article}
	\usepackage{Lapin}
	\newenvironment{Correction}{\par\tiny\blue\rule[1ex]{\textwidth}{1pt}\par\normalsize\textbf{\sffamily{}Correction de l'exercice~\theExo{} -- }}{\par\tiny\blue\rule[1ex]{\textwidth}{1pt}\par}
	\newtheorem{Exo}{{\sffamily{Exercice}}}
	\title{Tableaux et boucles}
	\author{Philippe \Nom{Rinaudo}\and{}Jean \Nom{Simard}}
	\date{\Date[l]{13}{10}{2009}}
\begin{document}
	\maketitle
	\section{Rappel}
		\subsection{Les tableaux}
			Un tableau va nous permettre de stocker plusieurs valeurs d'un même type dans un même ensemble.
			C'est une succession de variables.
			Par exemple, on pourrait utiliser un tableau pour y stocker \nombre{100}~nombres entiers à partir du crible d'Èratosthène.

			Pour commencer, il nous faut déclarer ce tableau.
			\begin{Code*}
unsigned int iDimension = 100;
unsigned int iEratosthene[iDimension];
			\end{Code*}
			Cette déclaration créé un tableau de \nombre{100}~valeurs de type \code{unsigned int}.
			Pour initialiser ce tableau, on va effectuer les opérations suivantes
			\begin{Code*}
iEratosthene[0] = 2;
iEratosthene[1] = 3;
iEratosthene[2] = 5;
iEratosthene[3] = 7;
iEratosthene[4] = 11;
...
			\end{Code*}
			On peut remarquer que la première case du tableau est numéroté zéro.
			Donc un tableau de \nombre{100}~valeurs est numérotée de \nombre{0} à \nombre{99}.
		\subsection{Les tableaux à plusieurs dimensions}
			Maintenant, imaginons que nous voulions stocker les valeurs d'une matrice.
			Prenons l'exemple d'une matrice $M$ telle que
			\begin{Equation}
				M = \left(
					\begin{array}{ccc}
						1 & 2 & 3 \\
						4 & 5 & 6 \\
						7 & 8 & 9
					\end{array}
					\right)
			\end{Equation}

			On peut bien évidemment, déclarer un tableau de \nombre{9}~valeurs et faire ensuite la calculs nécessaures pour accéder aux bonnes cases du tableau.
			Mais le langage \bsc{c} nous offre la possibilité de déclarer des tableaux à plusieurs dimensions.
			Pour cela on écrit
			\begin{Code*}
double dMatrice[3][3];
			\end{Code*}
			On peut alors remplir cette matrice avec
			\begin{Code*}
dMatrice[0][0] = 1.0;
dMatrice[0][1] = 2.0;
dMatrice[0][2] = 3.0;
dMatrice[1][0] = 4.0;
dMatrice[1][1] = 5.0;
dMatrice[1][2] = 6.0;
dMatrice[2][0] = 7.0;
dMatrice[2][1] = 8.0;
dMatrice[2][2] = 9.0;
			\end{Code*}
		\subsection{La boucle \code{for}}
			La boucle \code{for} va permettre d'effectuer un certains nombre d'itérations de boucle.
			En effet, pour l'instant, nous avons vu les boucles \code{while} et \code{do...while}.
			La boucle \code{for} possède deux différences avec les deux autres boucles déjà vues.
			Premièrement, la boucle \code{for} offre la possibilité de faire une initialisation avant de commencer la boucle.
			Deuxièmement, la boucle \code{for} donne la possibilité d'effectuer simplement une opération à chaque itération.

			La boucle \code{for} est souvent utilisée pour effectuer un nombre défini d'itérations.
			Elle va donc souvent être utilisée pour parcourir un tableau.
			Par exemple, affichons toutes les cases du tableau précédemment créé pour le crible d'Èrasthostène.
			L'initialisation de la boucle \code{for} sera la mise à zéro d'une variable \code{i}.
			Cette variable \code{i} sera incrémentée (ajouter \nombre{1}) pour parcourir les valeurs de \nombre{0} à \nombre{99}.
			Voici le code source
			\begin{Code*}
for( i = 0; i < 100; i++ )
{
	printf( "%f\n", iErathostene[i] );
}
			\end{Code*}
			
			\begin{Exo}
				Écrire un programme qui permet de calculer la somme des nombres de \nombre{0} à \nombre{100}.
			\end{Exo}
			\begin{Correction}
				\FichierCode{Somme.c}{cod-SommeC}
			\end{Correction}
			\begin{Exo}
				Remplir une matrice $5\times 5$ avec les valeurs de \nombre{1} à \nombre{25}.
			\end{Exo}
			\begin{Correction}
				\FichierCode{Matrice.c}{cod-MatriceC}
			\end{Correction}
			\begin{Exo}
				Écrire un programme qui permette de calculer la factorielle d'un nombre.
				L'utilisateur devra pouvoir choisir la valeur.
				Il faut vérifier que la valeur entrée par l'utilisateur est comprise dans l'intervalle $[0,15]$.
				
				\textbf{Rappel --} La factorielle d'un nombre se note $n!$ et est égal au produit des nombres entiers de l'intervalle $[1,n]$.
			\end{Exo}
			\begin{Correction}
				\FichierCode{Factorielle.c}{cod-FactorielleC}
			\end{Correction}
	\section{Approfondissement}
		\begin{Exo}
			Nous allons créer un petit jeu où l'objectif sera de deviner un nombre entre \nombre{0} et \nombre{100}.
			Pour que le jeu ait un intérêt, on va initialiser la solution avec une valeur aléatoire.
			Pour cela, il va falloir utiliser les bibliothèques \code{stdlib.h} et \code{time.h}.
			Puis pour initialiser la valeur de la solution, il faudra utiliser le code source suivant
			\begin{Code*}
int iSolution = 0;
srand( time( NULL ) );
iSolution = (int)( (double)rand() / ( (double)RAND_MAX + 1.0 ) * 100.0 );
			\end{Code*}
			L'utilisateur devra avoir cinq essais pour trouver la solution.
			À chaque essai le programme lui répondra si la solution est supérieure ou inférieure au nombre proposé.
		\end{Exo}
		\begin{Correction}
			\FichierCode{Jeu.c}{cod-JeuC}
		\end{Correction}
		\begin{Exo}
			Écrire un programme qui effectue la multiplication de deux matrices de dimension $5\times 5$.
			On réutilisera le code source précédant permettant d'initialiser la matrice.
		\end{Exo}
		\begin{Correction}
			\FichierCode{Multiplication.c}{cod-MultiplicationC}
		\end{Correction}
		\begin{Exo}
			Écrire un programme qui calcule le \bsc{pgcd} de deux nombres.

			\textbf{Rappel --} Le \bsc{pgcd} de deux nombres est le Plus Grand Commun Diviseur.
			C'est le plus grand nombre possible qui est un diviseur de chacun des deux nombres considérés.
			Par exemple, le \bsc{pgcd} de \nombre{51} et \nombre{34} est \nombre{17}.
			Le \bsc{pgcd} de \nombre{24} et \nombre{60} est \nombre{12}.
		\end{Exo}
		\begin{Correction}
			\FichierCode{PGCD.c}{cod-PGCDC}
		\end{Correction}
\end{document}
