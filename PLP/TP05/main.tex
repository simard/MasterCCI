\documentclass[a4paper]{article}
	\usepackage{Lapin}
	\newenvironment{Correction}{\par\tiny\blue\rule[1ex]{\textwidth}{1pt}\par\normalsize\textbf{\sffamily{}Correction de l'exercice~\theExo{} -- }}{\par\tiny\blue\rule[1ex]{\textwidth}{1pt}\par}
	\newtheorem{Exo}{{\sffamily{Exercice}}}
	\title{Boucles et chaînes de caractères}
	\author{Philippe \Nom{Rinaudo}\and{}Jean \Nom{Simard}}
	\date{\Date[l]{16}{10}{2009}}
\begin{document}
	\maketitle
	\section{Les chaînes de caractères}
		Nous allons travailler sur les chaînes de caractères et en profiter pour revenir un peu sur les boucles en utilisant plus particulièrement les boucles \code{for} que nous avons un peu laissé de côté jusqu'à maintenant.
		Une chaîne de caractère est un tableau de caractères (\code{char}).
		Mais ce tableau possède une particularité puisqu'il possède un caractère pour préciser où se trouve la fin.
		En effet, le dernier caractère d'une chaîne de caractères est le caractère \code{'\0'}.
		Le caractère \code{'\0'} n'est pas un caractère qu'on peut afficher ou visualiser car il fait parti des premiers caractères de la table \Nom{ascii} qui ne sont pas affichables.

		Par exemple, la chaîne de caractère \code{"Bonjour"} possède 8~caractères dont le huitième n'est pas visible \TabRef[v]{tab-Bonjour}.

		\begin{Table}{Chaîne de caractère \code{"Bonjour"}}{tab-Bonjour}{|l||c|c|c|c|c|c|c|c|}
			\hline
			Numéro & 1 & 2 & 3 & 4 & 5 & 6 & 7 & 8 \\
			\hline
			Caractère & \red\verb$'B'$ & \red\verb$'o'$ & \red\verb$'n'$ & \red\verb$'j'$ & \red\verb$'o'$ & \red\verb$'u'$ & \red\verb$'r'$ & \red\verb$'\0'$ \\
			\hline
			Affichage & \verb$B$ & \verb$o$ & \verb$n$ & \verb$j$ & \verb$o$ & \verb$u$ & \verb$r$ & \\
			\hline
		\end{Table}

		\begin{Exo}
			Écrire un programme qui déclare un tableau de 64~\code{char} initialisé avec la chaîne de caractères \code{"Vive les RERs"}.
			L'objectif est de lire cette chaîne de caractères avec une boucle \code{for} caractère par caractère puis d'afficher seulement ceux dont la valeur entière est inférieur ou égal à~90.

			Expliquer le résultat obtenu.
		\end{Exo}
		\begin{Correction}
			Nous parcourons la chaîne de caractères grâce à une boucle \code{for} qui aura comme condition d'arrêt le caractère \code{'\0'}.
			Puis nous les afficherons grâce à la fonction \code{printf} et du caractère de contrôle \code{%c}.
			\FichierCode{LectureMajuscules.c}{cod-LectureMajusculesC}
			Dans ce programme, nous affichons tous les caractères dont le code \Nom{ascii} est inférieur ou égal à~90.
			En faisant ceci, nous éliminons les lettres minuscules (entre~97 et~122) et conservont seulement les lettres majuscules qui se trouvent entre~65 et~90.
		\end{Correction}

	\section{La bibliothèque \code{string.h}}
		Pour utiliser aisément les chaînes de caractères, une bibliothèque existe nous proposant différentes fonctions.
		Par exemple, la fonction \code{strlen} permet de calculer la taille d'une chaîne de caractères.
		\code{strlen( "Bonjour" )} va renvoyer la valeur~7.
		\begin{Exo}
			Écrire un programme qui calcule la longueur d'une chaîne de caractères sans utiliser la fonction \code{strlen}.
			Pour vérifier votre résultat, comparer ce résultat à la fonction \code{strlen} et afficher «~\emph{C'est égal}~» ou «~\emph{C'est différent}~» suivant le cas.
		\end{Exo}
		\begin{Correction}
			Dans la correction de cet exercice, il faut utiliser une boucle.
			Ici, c'est une boucle \code{while} qui est utilisée mais les autres types de boucles peuvent aussi être utilisées.
			\FichierCode{Longueur.c}{cod-LongueurC}
		\end{Correction}

	\section{Les arguments de la fonction \code{main}}
		Nous allons utiliser pour la première fois les arguments de la fonction \code{main} qui sont \code{argc} et \code{argv}.
		Le paramètre \code{argc} est un paramètre qui contient le nombre d'arguments donnés au programme lors de son exécution.
		Le nom du programme constitue le premier argument.
		Par exemple, si on exécute le programme nommé \code{arguments}
		\begin{Code*}
$~$ ./arguments "Chaine assez longue" 8 25
		\end{Code*}
		le paramètre \code{argc} contient la valeur~4.
		Et c'est le paramètre \code{argv} qui contient les arguments sous forme d'un tableau de chaîne de caractères.
		C'est donc un tableau à deux dimensions de type \code{char}.
		Voici un tableau récapitulatif des arguments \TabRef{tab-RecapitulatifArguments}.
		\begin{Table}{Tableau récapitulatif des arguments}{tab-RecapitulatifArguments}{|l|l|}
			\hline
			Nom de l'argument & Valeur de l'argument \\
			\hline
			\hline
			\verb$argv[0]$ & \verb$arguments$ \\
			\hline
			\verb$argv[1]$ & \red\verb$"Chaine assez longue"$ \\
			\hline
			\verb$argv[2]$ & \verb$8$ \\
			\hline
			\verb$argv[3]$ & \verb$25$ \\
			\hline
		\end{Table}
		
		\textbf{Attention -- }Même si les arguments sont des chiffres, ils seront sauvegardés comme des caractères.
		\begin{Exo}
			Écrire un programme qui prend en entrée 3~paramètres.
			Le premier paramètre est une chaîne de caractères.
			Le second et le troisième paramètres sont des entiers compris entre~0 et~9.

			Le programme devra écrire seulement les caractères de la chaîne compris entre les deux entiers.
			Il faudra donc vérifier que le premier entier est inférieur au second.
			Voici quelques exemples d'exécution de ce programme \CodRef[v]{cod-ExempleExecution}.
			\FichierCode[caption={Exemple d'exécution},label={cod-ExempleExecution},numbers=none,language={}]{Arguments.txt}{cod-ArgumentsTXT}
		\end{Exo}
		\begin{Correction}
			Une petite astuce est nécessaire pour ce programme.
			En effet, nous lisons des paramètres qui sont des caractères hors nous devons utiliser des nombres.
			Étant donné que nous nous limitons aux valeurs entre~0 et~9 qui sont constitué de caractères simples, il nous faut donc convertir ces caractères en chiffres.
			Regardons dans la table \Nom{ascii}, les caractères de~0 à~9 se suivent dans cet ordre.
			Si on enlève la valeur entière correspondant au caractère \code{'0'} à chacun des caractère, on obtiendra la valeur désirée.
			Cette valeur est~48.
			Par exemple, enlevons~48 au caractère \code{'1'} (qui est égal à~49), on obtient bien le chiffre~1.
			N'oublions pas que cette opération est autorisée car les \code{char} sont des valeurs entières.
			De même, si on enlève~48 au caractère \code{'8'} qui a pour valeur décimale~56, on obtient la valeur~8.
			Cependant, on peut également enlever directement la valeur de \code{'0'} sans jamais écrire la valeur décimale correspondante.
			\FichierCode{Arguments.c}{cod-ArgumentsC}
		\end{Correction}

\end{document}
