\documentclass[a4paper]{article}
	\usepackage{Lapin}
	\newtheorem{Exo}{Exercice}
	\newenvironment{Correction}{\par\tiny\blue\rule[1ex]{\textwidth}{1pt}\par\normalsize\textbf{Correction de l'exercice~\theExo{} -- }}{\par\tiny\blue\rule[1ex]{\textwidth}{1pt}\par}
	\title{Fonctions}
	\author{Philippe \Nom{Rinaudo}\and{}Jean \Nom{Simard}}
	\date{\Date[l]{20}{10}{2009}}
\begin{document}
	\maketitle
	\section{Les fonctions}
		Nous allons enfin voir aujourd'hui comment déclarer et définir une fonction.
		En effet, une fonction doit tout d'abord être déclarée, comme une variable, puis ensuite elle est définie.
		De la même façon qu'une variable, une fonction possède un type.
		Ce type sera la nature de la valeur que la fonction renverra.
		Tous les types utilisés pour les variables peuvent être utilisés pour les fonctions (\code{int}, \code{float}, \code{char}\dots{}).
		Cependant, un type supplémentaire est disponible, le type \code{void}.
		Ce type permet d'écrire des fonctions qui ne renvoie pas de valeur.
		Ce type de fonction est appelé procédure.

		On peut par exemple créer une fonction qui affiche une matrice $\mathcal{M}_{3,3}$ qui ne renvoie pas de valeur.
		\FichierCode{AfficherMatrice.c}{cod-AfficherMatriceC}
		Ce genre de fonctions ne retournant pas de valeur, la clause \code{return} sans valeur peut être utilisé pour terminer la fonction.
		Le \CodRef{cod-AfficherMatriceC} est la définition de la fonction.
		La déclaration de cette même fonction est
		\begin{Code*}
void AfficherMatrice();
		\end{Code*}

		\begin{Exo}
			Nous allons aujourd'hui travailler sur les nombres complexes.
			L'objectif de cet exercice est de créer une structure pour les nombres complexes puis d'ajouter une fonction qui permette l'affichage de ces nombres complexes.
			Cette fonction d'affichage affichera une variable globale de type complexe.
			Dans la suite de ce TP, nous allons améliorer ce programme en évoluant la fonction et en ajoutant d'autres.
		\end{Exo}
		\begin{Correction}
			\FichierCode{AfficherComplexeVoid.c}{cod-AfficherComplexeVoidC}
		\end{Correction}
	\section{Les paramètres}
		À présent, nous allons introduire les paramètres d'une fonction.
		De la même façon que la fonction \code{main} possède les arguments \code{argc} et \code{argv}, nous allons ajouter de tels paramètres dans notre précédente fonction.
		Pour ajouter un paramètre, il suffit de le déclarer dans le nom de la fonction entre les parenthèses.
		
		Par exemple, on peut désirer afficher les nombres d'une matrice avec une précision de trois chiffres après la virgule.
		On aura donc recours à une fonction déclarée de la façon suivante
		\begin{Code*}
void AfficherMatrice( unsigned int iNbChiffres );
		\end{Code*}
		La ligne écrite ci-dessus constitue le prototype de la fonction.
		On appelle aussi cela la signature.
		La signature permet d'identifier n'importe quelle fonction par son nom, ses paramètres et la valeur qu'elle renvoie.
		\begin{Exo}
			Nous allons améliorer la fonction précédente d'affichage d'un complexe en ajoutant en paramètre un complexe.
			En effet, actuellement, la fonction peut être utilisée seulement pour le complexe global déclaré.
			Nous allons transformer la fonction pour qu'elle puisse être utilisée avec n'importe quel nombre complexe.
		\end{Exo}
		\begin{Correction}
			\FichierCode{AfficherComplexe.c}{cod-AfficherComplexeC}
		\end{Correction}
		\begin{Exo}
			Nous allons effectuer l'amélioration proposée pour l'affichage des matrices, c'est-à-dire ajouter un paramètre permettant de préciser la nombre de chiffres à afficher après la virgule.
			Pour cela, on peut déclarer un tableau de caractères qui contiendra le caractère de contrôle nécessaire à la fonction \code{printf}.
			Ensuite, au lieu de donner une chaîne de caractère à la fonction \code{printf}, on donnera le nom de ce tableau en paramètre.
		\end{Exo}
		\begin{Correction}
			\FichierCode{AfficherComplexeChiffre.c}{cod-AfficherComplexeChiffreC}
		\end{Correction}
	\section{La valeur renvoyée}
		Les fonctions permettent également de renvoyer une valeur comme nous l'avons déjà vu avec les fonctions \code{sqrt} \code{pow} par exemple ou même la fonction \code{main} dont nous n'utilisons pas la valeur.
		La valeur désirée peut donc être renvoyée grâce au mot-clef \code{return}.

		Par exemple, une fonction donnant la factorielle (que nous avons déjà vu) pourrait s'écrire
		\FichierCode{Factorielle.c}{cod-FactorielleC}
		\begin{Exo}
			Nous allons donc écrire quelques fonctions qui nous permettrons d'effectuer les différentes opérations de base sur les nombres complexes.
			Ajouter au programme d'affichage des nombres complexes, les fonctions suivantes :
			\begin{enumerate}
				\item Fonction qui renvoie le résultat de l'addition de deux nombres complexes;
				\item Fonction qui renvoie le résultat de la multiplication de deux nombres complexes;
				\item Fonction qui renvoie le conjugué d'un nombre complexe;
				\item Fonction qui renvoie la division d'un nombre complexe par un nombre réel simple;
				\item Fonction qui renvoie le résultat de la division de deux nombres complexes;
				\item Fonction qui affiche la forme polaire d'un nombre complexe (avec les unités en degrés).
			\end{enumerate}
		\end{Exo}
		\begin{Correction}
			\FichierCode{Complexes.c}{cod-ComplexesC}
		\end{Correction}
\end{document}
