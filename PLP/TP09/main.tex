\documentclass[a4paper]{article}
	\usepackage{Lapin}
	\newtheorem{Exo}{Exercice}
	\newenvironment{Correction}{\par\tiny\blue\rule[1ex]{\textwidth}{1pt}\par\normalsize\textbf{Correction de l'exercice~\theExo{} -- }}{\par\tiny\blue\rule[1ex]{\textwidth}{1pt}\par}
	\title{Algorithmes de tri}
	\author{Jean \Nom{Simard}}
	\date{\Date[l]{04}{11}{2009}}
\begin{document}
	\maketitle
	\section{Préalable}
		Nous allons mettre en place une base de programme sur laquelle nous allons tester nos fonctions de tri.
		Chaque exercice sera une amélioration apportée à ce qui a déjà été fait.
		C'est pour cette raison que vous ne devriez avoir qu'un seul fichier source pour l'ensemble des exercices de ce TP.

		Nous allons travailler sur des listes doublement chaînées.
		Chaque maillon de cette liste sera constitué d'un nombre (puis des pointeurs vers le maillon précédent et suivant).
		\begin{Exo}
			Écrire la structure qui remplira la tâche de maillon d'une liste doublement chaînée et contenant un entier non-signé.
		\end{Exo}
		\begin{Correction}
			\FichierCode[firstline=5,lastline=11]{Tri.c}{cod-TriCStructure}
		\end{Correction}

		Dans notre programme, nous allons devoir initialiser la liste doublement chaînée avec des valeurs.
		Ces valeurs doivent être variées pour que le tri ait un intérêt.
		\begin{Exo}
			Écrire une fonction qui créé et initialise les différents maillons de chaîne doublement chaînée.
			Cette fonction sera appelée depuis la fonction \code{main} avec un paramètre indiquant le nombre de maillons à créer.
			Le nombre de maillons à créer sera une valeur que l'utilisateur donnera (au choix, par argument ou par l'intermédiaire de la fonction \code{scanf}).
			Les maillons seront initialisés avec un entier aléatoire entre zéro et le nombre de maillon.
		\end{Exo}
		\begin{Correction}
			\FichierCode[firstline=12,lastline=33]{Tri.c}{cod-TriCInitialiserListe}
		\end{Correction}
		\begin{Exo}
			Écrire une fonction qui libère l'espace mémoire de cette liste doublement chaînée.
			Cette fonction sera appelée juste avant de quitter la fonction \code{main}.
		\end{Exo}
		\begin{Correction}
			\FichierCode[firstline=34,lastline=44]{Tri.c}{cod-TriCLibererListe}
		\end{Correction}
		\begin{Exo}
			Écrire une fonction qui permettre l'affichage des valeurs entières non-signées de la liste doublement chaînées.
			Cette fonction sera utilisée pour afficher la liste aléatoirement créée puis la liste triées.
		\end{Exo}
		\begin{Correction}
			\FichierCode[firstline=45,lastline=55]{Tri.c}{cod-TriCAfficherListe}
		\end{Correction}
		
	\section{Tri à bulles}
		Pour rappel, l'algorithme de tri à bulles consiste à parcourir la liste en comparant deux à deux les valeurs puis à les inverser si elles ne sont pas dans le bon ordre.
		On effectue cette opération jusqu'à atteindre la fin de la liste.
		On parcours de nouveau la liste tant que la liste n'est pas triée.
		\begin{Exo}
			Écrire une fonction qui effectue un tri à bulle de la liste doublement chaînée.
		\end{Exo}
		\begin{Correction}
			\FichierCode[firstline=56,lastline=126]{Tri.c}{cod-TriCTriBulle}
		\end{Correction}
	\section{Tri par sélection}
		Le tri par sélection consiste à chercher le maillon dont la valeur est la plus petite puis à le placer au début de la liste.
		Ensuite, on effectue la même opération sur la liste sur le reste de la liste qui n'est pas encore trié.
		\begin{Exo}
			Écrire une fonction qui effectue un tri par sélection de la liste doublement chaînée.
		\end{Exo}
		\begin{Correction}
			\FichierCode[firstline=127,lastline=153]{Tri.c}{cod-TriCTriSelection}
		\end{Correction}
	\section{Tri par insertion}
		Le tri par insertion va le plus souvent être plus efficace que les deux algorithmes précédents.
		Il va parcourir la liste est lorsqu'il trouve un élément qui n'est pas à la bonne place, il va le ramener dans la liste en l'insérant à sa place.
		\begin{Exo}
			Écrire une fonction qui effectue un tri par insertion de la liste doublement chaînée.
		\end{Exo}
		\begin{Correction}
			\FichierCode[firstline=154,lastline=204]{Tri.c}{cod-TriCTriInsertion}
		\end{Correction}
\end{document}
